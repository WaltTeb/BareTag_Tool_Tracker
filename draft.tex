\documentclass[conference]{IEEEtran}
\IEEEoverridecommandlockouts
% The preceding line is only needed to identify funding in the first footnote. If that is unneeded, please comment it out.
\usepackage{cite}
\usepackage{amsmath,amssymb,amsfonts}
\usepackage{algorithmic}
\usepackage{graphicx}
\usepackage{textcomp}
\usepackage{xcolor}
\def\BibTeX{{\rm B\kern-.05em{\sc i\kern-.025em b}\kern-.08em
    T\kern-.1667em\lower.7ex\hbox{E}\kern-.125emX}}
\begin{document}

\title{BareTag Tool-Tracker*\\
{\footnotesize \textsuperscript{*}Note: Sub-titles are not captured in Xplore and
should not be used}
\thanks{Identify applicable funding agency here. If none, delete this.}
}

\author{\IEEEauthorblockN{Walter Tebbetts}
\textit{CompE}\\
\textit{Software Lead}

\and
\IEEEauthorblockN{Sean Brown}
\textit{CompE}\\
\textit{Logistics Lead}

\and
\IEEEauthorblockN{Ken Su}
\textit{CompE}\\
\textit{Budget Lead}

\and
\IEEEauthorblockN{Connor McGarry}
\textit{CompE}\\
\textit{Hardware Lead}

}

\maketitle

\begin{abstract}
    Over the past 10 years, the majority of tools on a construction 
    site have converted from wired to battery-powered. While this 
    makes tools easy to move around, it also makes them easy to misplace
    and an easier target for theives. Data indicates that the construction
    idustry suffered nearly \$1,000,000,000 in loses due to tool theft in
    2023 alone [1], strongly indicating the need for a robust and effective 
    theft mitigation system. We propose the BareTag Tool-Tracker, a novel
    approach to tool tracking that utilizes Ultra-Wideband(UWB) and Bluetooth 
    Low-Energy(BLE) radio in order to real-time track tools, materials, or other 
    valuable items on a construction site. The system utilizes a series
    of pre-placed Anchor posts, that send UWB pings to a Tag that is connected
    to a tool. Each Anchor can then calculate its distance to a Tag, relaying
    that information to a Base station over Long Range(LoRa) radio. The 
    base station runs the aggregated distance data through a multilateration
    algorithm that can calculate the Tag's location with $\pm$ 10 cm accuracy.
    The calculated location is then output to a local terminal, as well as 
    uploaded to a cloud database for future reference. Altogether, the BareTag Tool-Tracker
    is highly accurate ($\pm$ 10 cm), low-power (1 year of battery life), and
    scalable (increase range by adding additional Anchors).
\end{abstract}

\begin{IEEEkeywords}
component, formatting, style, styling, insert
\end{IEEEkeywords}

\section{Introduction}
On constructions sites tools are constantly being misplaced or passed around causing workers to spend extra time searching and waiting before returning to their assignments. Furthermore, in recent years research has shown that theft on constructions sites has built up costs for construction companies and slowing down projects, very detrimental for small companies. In recent years many different approaches have been taken to solve this growing issue. 
\subsection{Significance}
In 2016 it was estimated that in the United States alone \$1,000,000,000 worth of construction tools were stolen [1]. A survey by the Charted Institute of Building discovered that out of the 1000 construction mangers interviewed, a third responded that they had experienced theft weekly on their sites. It was estimated that each of these weekly incidents cost the business an average of \$6,000, in some cases, in a single night the site had lost \$100,000 worth of equipment [2]. What’s worse is that this is a growing case. The FBI has reported that in 2021, theft on construction sites had outgrown theft in convenience stores [3]. Many managers have reported that these incidents have escalated to organized crime with evidence of sophisticated planning and coordinated executions. [2]. Theft on the site is not only costly to the business owners but also even more inconvenient for the construction workers and their managers. Due to the lack of proper tools construction workers may not continue with their assignments and managers have to keep pushing deadlines. The result of this costs the business not just in extra wages, but also extra insurance and the company’s reputation [2]. 

\subsection{Context and Survey of Similar Solutions}

The IEEEtran class file is used to format your paper and style the text. All margins, 
column widths, line spaces, and text fonts are prescribed; please do not 
alter them. You may note peculiarities. For example, the head margin
measures proportionately more than is customary. This measurement 
and others are deliberate, using specifications that anticipate your paper 
as one part of the entire proceedings, and not as an independent document. 
Please do not revise any of the current designations.

\subsection{Societal Impacts}
Define abbreviations and acronyms the first time they are used in the text, 
even after they have been defined in the abstract. Abbreviations such as 
IEEE, SI, MKS, CGS, ac, dc, and rms do not have to be defined. Do not use 
abbreviations in the title or heads unless they are unavoidable.

\subsection{Goals Specifications and Testing Plan}
\begin{itemize}
\item Use either SI (MKS) or CGS as primary units. (SI units are encouraged.) English units may be used as secondary units (in parentheses). An exception would be the use of English units as identifiers in trade, such as ``3.5-inch disk drive''.
\item Avoid combining SI and CGS units, such as current in amperes and magnetic field in oersteds. This often leads to confusion because equations do not balance dimensionally. If you must use mixed units, clearly state the units for each quantity that you use in an equation.
\item Do not mix complete spellings and abbreviations of units: ``Wb/m\textsuperscript{2}'' or ``webers per square meter'', not ``webers/m\textsuperscript{2}''. Spell out units when they appear in text: ``. . . a few henries'', not ``. . . a few H''.
\item Use a zero before decimal points: ``0.25'', not ``.25''. Use ``cm\textsuperscript{3}'', not ``cc''.)
\end{itemize}


\section{Design}

\subsection{Overview}
At a high level, the BareTag Tool Tracker utilizes Ultra-Wideband (UWB), 
Bluetooth Low Energy (BLE), and LoRa radio in order to real-time track an 
item's location on a construction site. The technology at the core of our 
design is UWB radio. UWB radio is a form of radio communication that 
utilizes pulses of radio energy at specifically timed intervals in order 
to transmit information. This protocol is not ideal for data communication, 
but is very accurate in performing distance ranging. With two UWB 
transceivers, one configured as the controller (Anchor) and the other 
configured as the responder (Tag), the controller can send a UWB ping to 
the responder. The responder will almost immediately send back a response 
ping, the controller can then use the time between when it sent its ping to 
when it received the response ping, in order to calculate the distance 
between the two transceivers.

\[ d = \frac{\frac{ToF}{2}}{c} \]
Where $d$ is distance, $ToF$ is time of flight, and $c$ is the speed of light.


\subsection{Tag}

\subsection{Block 2}

\subsection{Block 3}

\subsection{Block 4}


\section*{Acknowledgment}

The preferred spelling of the word ``acknowledgment'' in America is without 
an ``e'' after the ``g''. Avoid the stilted expression ``one of us (R. B. 
G.) thanks $\ldots$''. Instead, try ``R. B. G. thanks$\ldots$''. Put sponsor 
acknowledgments in the unnumbered footnote on the first page.

\section*{References}

Please number citations consecutively within brackets \cite{b1}. The 
sentence punctuation follows the bracket \cite{b2}. Refer simply to the reference 
number, as in \cite{b3}---do not use ``Ref. \cite{b3}'' or ``reference \cite{b3}'' except at 
the beginning of a sentence: ``Reference \cite{b3} was the first $\ldots$''

Number footnotes separately in superscripts. Place the actual footnote at 
the bottom of the column in which it was cited. Do not put footnotes in the 
abstract or reference list. Use letters for table footnotes.

Unless there are six authors or more give all authors' names; do not use 
``et al.''. Papers that have not been published, even if they have been 
submitted for publication, should be cited as ``unpublished'' \cite{b4}. Papers 
that have been accepted for publication should be cited as ``in press'' \cite{b5}. 
Capitalize only the first word in a paper title, except for proper nouns and 
element symbols.

For papers published in translation journals, please give the English 
citation first, followed by the original foreign-language citation \cite{b6}.

\begin{thebibliography}{00}
\bibitem{b1} G. Eason, B. Noble, and I. N. Sneddon, ``On certain integrals of Lipschitz-Hankel type involving products of Bessel functions,'' Phil. Trans. Roy. Soc. London, vol. A247, pp. 529--551, April 1955.
\bibitem{b2} J. Clerk Maxwell, A Treatise on Electricity and Magnetism, 3rd ed., vol. 2. Oxford: Clarendon, 1892, pp.68--73.
\bibitem{b3} I. S. Jacobs and C. P. Bean, ``Fine particles, thin films and exchange anisotropy,'' in Magnetism, vol. III, G. T. Rado and H. Suhl, Eds. New York: Academic, 1963, pp. 271--350.
\bibitem{b4} K. Elissa, ``Title of paper if known,'' unpublished.
\bibitem{b5} R. Nicole, ``Title of paper with only first word capitalized,'' J. Name Stand. Abbrev., in press.
\bibitem{b6} Y. Yorozu, M. Hirano, K. Oka, and Y. Tagawa, ``Electron spectroscopy studies on magneto-optical media and plastic substrate interface,'' IEEE Transl. J. Magn. Japan, vol. 2, pp. 740--741, August 1987 [Digests 9th Annual Conf. Magnetics Japan, p. 301, 1982].
\bibitem{b7} M. Young, The Technical Writer's Handbook. Mill Valley, CA: University Science, 1989.
\end{thebibliography}
\vspace{12pt}
\color{red}
IEEE conference templates contain guidance text for composing and formatting conference papers. Please ensure that all template text is removed from your conference paper prior to submission to the conference. Failure to remove the template text from your paper may result in your paper not being published.

\end{document}
